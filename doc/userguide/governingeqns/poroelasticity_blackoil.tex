% ----------------------------------------------------------------------
\section{Multiphase Poromechanics with Infinitesimal Strain and No Faults}

In order to expand our poroelastic model to account for cases such as oil reservoirs
 or coastal aquifers, where multiple fluids typically exist simultaneously, representation
 of multiple fluid phases are required. For an initial example of such behavior,
 the black oil model, traditionally used to model low volatility oil system, is
 applied.


\subsection{Solubility}
A reservoir fluid may contain multiple components. Say a sample of fluid contains 
both oil and gas components at reservoir conditions, of masses $m_{o}$, and $m_{g}$,
respectively. The volume of these components when measured at standard(surface) 
conditions may be defined as:
\begin{equation}
V_{oS} = \frac{m_{o}}{\rho_{oS}}, \quad V_{gS} = \frac{m_{g}}{\rho_{gS}},
\end{equation}
 
with $\rho_{oS}$ and $\rho_{gS}$ representing the oil and gas densities when measured
at surface conditions. To account for pressure dependent solubility of gas in oil, the
gas-oil solubility ratio, $R_{s}$, may be defined as the volume of gas dissolved at reservoir
pressure per unit volume of oil, when measured at standard conditions:

\begin{equation}
  R_{s} = \frac{\left(V_{gS}\right)_{dissolved}}{V_{oS}} = \frac{\left(m_{g}\right)_{dissolved} \rho_{oS}}{m_{o}\rho_{gS}}.
\end{equation}

Pressure dependent changes in fluid densities are accounted for through fluid formation volume factors.
In this example, the volume of the oil phase is larger at reservoir conditions than at standard conditions
due to the dissolved gas at reservoir pressure. For our two phase example, the oil formation volume
factor, $B_{o}$, is defined as the ratio of the volume of oil and dissolved gas at reservoir conditions
to the volume of oil measured at standard conditions,
\begin{equation}
  B_{o} = \frac{V_{o}}{V_{oS}},
\end{equation}

and

\begin{equation}
  V_{o} = \frac{\left(m_{o} + \left(m_{g}\right)_{dissolved}\right)}{\rho_{o}},
\end{equation}

where $\rho_{o}$ represents the density of oil at reservoir conditions. This results in:

\begin{equation}
  B_{o} = \frac{\left(m_{o} + \left(m_{g}\right)_{dissolved} \right) \rho_{oS}}{m_{o}\rho_{o}}.
\end{equation}

The formation volume factor for gas, $B_{g}$, is defined here as the ratio of the volume of free (not dissolved)
gas at reservoir conditions to the volume of the same gas at standard conditions, or

\begin{equation}
  B_{g} = \frac{V_{g}}{V_{gS}}
\end{equation}



\subsection{Interfacial Tension and Capillary Pressure}


\subsection{Relative Permeability}












We base this formulation for poroelsticity on Zheng et al. and
Detournay and Cheng (1993). We assume a slightly compressible fluid
that completely saturates a porous solid, undergoing infinitesimal
strain.

We begin with the conservation of linear momentum, including inertia,
borrowed from linear elasticity:
\begin{equation}
    \rho_s\frac{\partial^2 \vec{u}}{\partial t^2} = \vec{f}(t) + \nabla \cdot \tensor{\sigma}(\vec{u},p).
\end{equation}
Enforcing mass balance of the fluid gives
\begin{gather}
  \frac{\partial \zeta(\vec{u},p)}{\partial t} + \nabla \cdot \vec{q}(p) =
  \gamma(\vec{x},t) \text{ in } \Omega, \\
%
  \vec{q} \cdot \vec{n} = q_0(\vec{x},t) \text{ on }\Gamma_q, \\
%
  p = p_0(\vec{x},t) \text{ on }\Gamma_p,
\end{gather}
where $\zeta$ is the variation in fluid content, $\vec{q}$ is the rate
of fluid volume crossing a unit area of the porous solid, $\gamma$ is
the rate of injected fluid per unit volume of the porous solid, $q_0$
is the outward fluid velocity normal to the boundary $\Gamma_q$, and
$p_0$ is the fluid pressure on boundary $\Gamma_p$.

We require the fluid flow to follow Darcy's law (Navier-Stokes
equation neglecting inertial effects),
\begin{equation}
  \vec{q}(p) = -\frac{\tensor{k}}{\mu_{f}}(\nabla p - \vec{f}_f),
\end{equation}
where $\tensor{k}$ is the intrinsic permeability, $\mu_f$ is the viscosity of the
fluid, $p$ is the fluid pressure, and $\vec{f}_f$ is the body force
in the fluid. If gravity is included in a problem, then usually
$\vec{f}_f = \rho_f \vec{g}$, where $\rho_f$ is the density of the
fluid and $\vec{g}$ is the gravitational acceleration vector.

\subsection{Constitutive Behavior}

We assume linear elasticity for the solid phase, so the constitutive behavior can be expressed
as
\begin{equation}
  \tensor{\sigma}(\vec{u},p) = \tensor{C} : \tensor{\epsilon} - \alpha p \tensor{I},
\end{equation}
where $\tensor{\sigma}$ is the stress tensor, $\tensor{C}$ is the
tensor of elasticity constants, $\alpha$ is the Biot coefficient
(effective stress coefficient), $\tensor{\epsilon}$ is the strain
tensor, and $\tensor{I}$ is the identity tensor.  For this case, we
will assume that the material properties are isotropic, resulting in
the following formulation for the stress tensor:
\begin{equation}
    \tensor{\sigma}(\vec{u},p) = \tensor{C}:\tensor{\epsilon} - \alpha p \tensor{I}
                                           = \lambda \tensor{I} \epsilon_{v} + 2 \mu - \alpha \tensor{I} p
\end{equation}
where $\lambda$ and $\mu$ are Lam\'e's parameters,
$\lambda = K_{d} - \frac{2 \mu}{3}$, $\mu$ is the shear modulus, and
the volumetric strain is defined as
$\epsilon_{v} = \nabla \cdot \vec{u}$.

For the constitutive behavior of the fluid, we use the volumetric
strain to couple the fluid-solid behavior,
\begin{gather}
  \zeta(\vec{u},p) = \alpha \Tr({\tensor{\epsilon}}) + \frac{p}{M}, \\
%
  \frac{1}{M} = \frac{\alpha-\phi}{K_s} + \frac{\phi}{K_f},
\end{gather}
where $1/M$ is the specific storage coefficient at constant strain,
$K_s$ is the bulk modulus of the solid, and $K_f$ is the bulk modulus
of the fluid. We can write the trace of the strain tensor as the dot
product of the gradient and displacement field, so we have
\begin{equation}
  \zeta(\vec{u},p) = \alpha (\nabla \cdot \vec{u}) + \frac{p}{M}.
\end{equation}

\begin{table}[htbp]
  \caption{Mathematical notation for poroelasticity with
    infinitesimal strain.}
  \label{tab:notation:poroelasticity}
  \begin{tabular}{lcp{3.5in}}
    \toprule
    {\bf Category} & {\bf Symbol} & {\bf Description} \\
    \midrule
    Unknowns           & $\vec{u}$ & Displacement field \\
                       & $\vec{v}$ & Velocity field \\
                       & $p$       & Pressure field (corresponds to pore fluid pressure) \\
                       & $\epsilon_{v}$ & Volumetric (trace) strain \\
    \hline
    Derived quantities & $\tensor{\sigma}$ & Cauchy stress tensor \\
                       & $\tensor{\epsilon}$ & Cauchy strain tensor \\
                       & $\zeta$ & Variation of fluid content (variation of fluid vol. per unit vol. of PM), $\alpha \epsilon_{v} + \frac{p}{M}$ \\
                       & $\rho_{b}$ & Bulk density, $\left(1 - \phi\right) \rho_{s} + \phi \rho_{f}$ \\
                       & $\vec{q}$ & Darcy flux, $-\frac{\tensor{k}}{\mu_{f}} \cdot \left(\nabla p - \vec{f}_{f}\right)$ \\
                       & $M$ & Biot Modulus, $\frac{K_{f}}{\phi} + \frac{K_{s}}{\alpha - \phi}$ \\
    \hline
    Common constitutive parameters & $\rho_{f}$ & Fluid density \\
                       & $\rho_{s}$ & Solid (matrix) density \\
                       & $\phi$ & Porosity \\
                       & $\tensor{k}$ & Permeability \\
                       & $\mu_{f}$ & Fluid viscosity \\
                       & $K_{s}$ & Solid grain bulk modulus \\
                       & $K_{f}$ & Fluid bulk modulus \\
                       & $K_{d}$ & Drained bulk modulus \\
                       & $\alpha$ & Biot coefficient, $1 - \frac{K_{d}}{K_{s}}$ \\
    \hline
    Source terms       & $\vec{f}$ & Body force per unit volume, for example: $\rho_{b} \vec{g}$ \\
                       & $\vec{f}_{f}$ & Fluid body force, for example: $\rho_{f} \vec{g}$ \\
                       & $\gamma$ & Source density; rate of injected fluid per unit volume of the porous solid \\
    \bottomrule
  \end{tabular}
\end{table}
